\documentclass[11pt]{article}

% ---------- Packages ----------
\usepackage{geometry}
\geometry{letterpaper, margin=1in}
\usepackage{times}
\usepackage{graphicx}
\usepackage{listings}
\usepackage{xcolor}
\usepackage[colorlinks=true, linkcolor=blue, urlcolor=blue]{hyperref}
\usepackage{caption}
\usepackage{booktabs}
\usepackage{amsmath}
\usepackage{longtable}
\usepackage{adjustbox}


% ---------- Document ----------
\begin{document}

\title{CMPUT 692 -- Topics in Data Management with LLMs \\ 
Assignment 1}
\author{Daniel Penner \\ 1908155}
\date{\today}
\maketitle

All associated code for this assignment can be found on \href{https://github.com/Daniel-Penner/CMPUT-692-Assignment-1}{\underline{my Github}}.

\section{Query Selection}

\subsection{Query List}
\begin{longtable}{l l l}
\toprule
Query ID & Database & Difficulty \\
\midrule
91   & financial            & Simple \\
96   & financial            & Simple \\
101  & financial            & Simple \\
108  & financial            & Simple \\
117  & financial            & Moderate \\
128  & financial            & Moderate \\
136  & financial            & Moderate \\
149  & financial            & Challenging \\
169  & financial            & Challenging \\
173  & financial            & Challenging \\
393  & card\_games          & Simple \\
394  & card\_games          & Simple \\
406  & card\_games          & Simple \\
411  & card\_games          & Simple \\
427  & card\_games          & Moderate \\
432  & card\_games          & Moderate \\
434  & card\_games          & Moderate \\
477  & card\_games          & Challenging \\
507  & card\_games          & Challenging \\
513  & card\_games          & Challenging \\
547  & codebase\_community  & Simple \\
550  & codebase\_community  & Simple \\
556  & codebase\_community  & Simple \\
558  & codebase\_community  & Simple \\
572  & codebase\_community  & Moderate \\
578  & codebase\_community  & Moderate \\
587  & codebase\_community  & Moderate \\
598  & codebase\_community  & Challenging \\
639  & codebase\_community  & Challenging \\
701  & codebase\_community  & Challenging \\
789  & superhero            & Simple \\
793  & superhero            & Simple \\
795  & superhero            & Simple \\
806  & superhero            & Simple \\
814  & superhero            & Moderate \\
817  & superhero            & Moderate \\
823  & superhero            & Moderate \\
829  & superhero            & Challenging \\
834  & superhero            & Challenging \\
835  & superhero            & Challenging \\
1470 & debit\_card\_specializing & Simple \\
1477 & debit\_card\_specializing & Simple \\
1476 & debit\_card\_specializing & Challenging \\
1478 & debit\_card\_specializing & Simple \\
1486 & debit\_card\_specializing & Simple \\
1490 & debit\_card\_specializing & Moderate \\
1501 & debit\_card\_specializing & Moderate \\
1516 & debit\_card\_specializing & Moderate \\
1526 & debit\_card\_specializing & Challenging \\
1531 & debit\_card\_specializing & Moderate \\
\bottomrule
\caption{The selected subset of queries.}
\end{longtable}

\subsection{Justification}
The 5 categories were randomly selected. For each category 10 queries were chosen: 4 simple, 3 moderate, 3 challenging (with the exception of "debit card specializing" which only had 2 challenging).

\section{Query Generation}
\subsection{Generated SQL List}

Below is the full set of generated SQL queries for the selected natural language questions. Each query is referenced by its \texttt{question\_id}.

\begin{enumerate}

\item[91:]
\begin{lstlisting}[language=SQL]
SELECT CASE WHEN AVG(district.A6)>AVG(district.A7) THEN '1995'
            WHEN AVG(district.A6)<AVG(district.A7) THEN '1996'
            ELSE 'tie' END AS higher_year,
       AVG(district.A6) AS avg_1995,
       AVG(district.A7) AS avg_1996
FROM district;
\end{lstlisting}

\item[96:]
\begin{lstlisting}[language=SQL]
SELECT COUNT(DISTINCT disp.client_id)
FROM disp
JOIN account ON disp.account_id = account.account_id
WHERE UPPER(disp.type) = 'OWNER'
  AND UPPER(account.frequency) LIKE '%WEEK%';
\end{lstlisting}

\item[101:]
\begin{lstlisting}[language=SQL]
SELECT DISTINCT account.account_id
FROM account
JOIN trans ON account.account_id = trans.account_id
WHERE substr(trans.date,1,4)='1995'
  AND trans.date = (SELECT MIN(date)
                    FROM trans
                    WHERE substr(date,1,4)='1995');
\end{lstlisting}

\item[108:]
\begin{lstlisting}[language=SQL]
SELECT t.amount
FROM trans t
JOIN account a ON t.account_id=a.account_id
JOIN loan l ON l.account_id=a.account_id
WHERE l.amount=(SELECT MAX(amount) FROM loan)
  AND t.date>=a.date
ORDER BY t.date
LIMIT 1;
\end{lstlisting}

\item[117:]
\begin{lstlisting}[language=SQL]
SELECT COALESCE(
  100.0 * SUM(CASE WHEN loan.status = 'closed' THEN loan.amount ELSE 0 END) 
  / NULLIF(SUM(loan.amount),0), 0)
FROM loan;
\end{lstlisting}

\item[128:]
\begin{lstlisting}[language=SQL]
SELECT district.A2, COUNT(DISTINCT client.client_id) AS cnt
FROM client
JOIN disp ON client.client_id=disp.client_id
JOIN account ON disp.account_id=account.account_id
JOIN district ON account.district_id=district.district_id
WHERE client.gender='F' AND disp.type='OWNER'
GROUP BY district.A2
ORDER BY cnt DESC
LIMIT 9;
\end{lstlisting}

\item[136:]
\begin{lstlisting}[language=SQL]
SELECT loan.account_id, COUNT(*)
FROM loan
JOIN account ON loan.account_id=account.account_id
WHERE loan.date BETWEEN '1995-01-01' AND '1997-12-31'
  AND loan.amount>=250000
  AND account.frequency='M'
  AND loan.status='approved'
GROUP BY loan.account_id;
\end{lstlisting}

\item[149:]
\begin{lstlisting}[language=SQL]
SELECT DISTINCT account.frequency
FROM account
JOIN district ON account.district_id = district.district_id
WHERE district.A11>8000 AND district.A11<=9000
  AND account.account_id NOT IN (
      SELECT account_id FROM loan
  );
\end{lstlisting}

\item[169:]
\begin{adjustbox}{max width=\textwidth}
\begin{lstlisting}[language=SQL]
SELECT CASE WHEN t.s1996=0 THEN NULL
            ELSE (t.s1997 - t.s1996)*1.0 / t.s1996 END AS growth
FROM (
  SELECT SUM(CASE WHEN strftime('%Y',loan.date)='1996' THEN loan.amount ELSE 0 END) AS s1996,
         SUM(CASE WHEN strftime('%Y',loan.date)='1997' THEN loan.amount ELSE 0 END) AS s1997
  FROM loan
  JOIN disp ON loan.account_id=disp.account_id
  JOIN client ON disp.client_id=client.client_id
  WHERE client.gender='M'
) t;
\end{lstlisting}
\end{adjustbox}

\item[173:]
\begin{lstlisting}[language=SQL]
SELECT account.frequency,
       (SELECT trans.k_symbol
        FROM trans
        WHERE trans.account_id=3 AND trans.type='Debit'
        GROUP BY trans.k_symbol
        HAVING SUM(trans.amount)=3539)
FROM account
WHERE account.account_id=3;
\end{lstlisting}

\item[393:]
\begin{lstlisting}[language=SQL]
SELECT COUNT(*)
FROM cards
WHERE artist = 'John Avon' AND hasFoil = 0;
\end{lstlisting}

\item[394:]
\begin{lstlisting}[language=SQL]
SELECT COUNT(*)
FROM cards
WHERE borderColor = 'white' AND CAST(power AS INTEGER) >= 5;
\end{lstlisting}

\item[406:]
\begin{lstlisting}[language=SQL]
SELECT DISTINCT cards.id
FROM cards
JOIN legalities ON cards.uuid = legalities.uuid
WHERE cards.type LIKE '%Creature%';
\end{lstlisting}

\item[411:]
\begin{lstlisting}[language=SQL]
SELECT cards.artist
FROM cards
WHERE cards.text = 'Das perfekte Gegenmittel zu einer dichten Formation';
\end{lstlisting}

\item[427:]
\begin{lstlisting}[language=SQL]
SELECT DISTINCT language
FROM set_translations
WHERE setCode = 'ARC';
\end{lstlisting}

\item[432:]
\begin{lstlisting}[language=SQL]
SELECT sets.name
FROM sets
JOIN set_translations ON set_translations.setCode = sets.code
WHERE set_translations.language = 'Russian'
ORDER BY sets.totalSetSize DESC
LIMIT 1;
\end{lstlisting}

\item[434:]
\begin{lstlisting}[language=SQL]
SELECT COUNT(*)
FROM sets s
WHERE s.isOnlineOnly = 0
  AND EXISTS (
    SELECT 1 FROM set_translations st
    WHERE st.setCode = s.code AND st.language = 'Japanese'
  )
  AND NOT EXISTS (
    SELECT 1 FROM set_translations st2
    WHERE st2.setCode = s.code AND st2.language != 'Japanese'
  );
\end{lstlisting}

\item[477:]
\begin{lstlisting}[language=SQL]
SELECT DISTINCT cards.artist
FROM cards
JOIN sets ON cards.setCode = sets.code
WHERE sets.name = 'Coldsnap'
  AND cards.artist IN ('Jeremy Jarvis','Aaron Miller','Chippy');
\end{lstlisting}

\item[507:]
\begin{lstlisting}[language=SQL]
SELECT 100.0 * SUM(isOnlineOnly) / COUNT(*) AS percentage
FROM (
  SELECT DISTINCT sets.code, sets.isOnlineOnly
  FROM sets
  JOIN set_translations ON sets.code = set_translations.setCode
  WHERE set_translations.language = 'Portuguese (Brazil)'
);
\end{lstlisting}

\item[513:]
\begin{lstlisting}[language=SQL]
SELECT id
FROM sets
WHERE type = 'Commander'
ORDER BY totalSetSize DESC
LIMIT 1;
\end{lstlisting}

\item[547:]
\begin{lstlisting}[language=SQL]
SELECT COUNT(*)
FROM posts
JOIN users ON posts.OwnerUserId = users.Id
WHERE users.Age >= 60 AND posts.Score > 19;
\end{lstlisting}

\item[550:]
\begin{lstlisting}[language=SQL]
SELECT posts.Body
FROM posts
JOIN tags ON tags.ExcerptPostId = posts.Id
ORDER BY tags.Count DESC
LIMIT 1;
\end{lstlisting}

\item[556:]
\begin{lstlisting}[language=SQL]
SELECT AVG(badge_count)
FROM (
  SELECT users.Id, COUNT(badges.Id) AS badge_count
  FROM users
  LEFT JOIN badges ON users.Id = badges.UserId
  WHERE users.Views > 200
  GROUP BY users.Id
);
\end{lstlisting}

\item[558:]
\begin{lstlisting}[language=SQL]
SELECT COUNT(*)
FROM votes
JOIN posts ON votes.PostId = posts.Id
WHERE posts.OwnerUserId = 58
  AND date(votes.CreationDate) = '2010-07-19';
\end{lstlisting}

\item[572:]
\begin{adjustbox}{max width=\textwidth}
\begin{lstlisting}[language=SQL]
SELECT posts.ViewCount
FROM posts
WHERE posts.Title = 'Integration of Weka and/or RapidMiner into Informatica PowerCenter/Developer';
\end{lstlisting}
\end{adjustbox}

\item[578:]
\begin{lstlisting}[language=SQL]
SELECT users.DisplayName, users.Reputation
FROM posts
JOIN users ON posts.OwnerUserId = users.Id
WHERE posts.Title = 'Understanding what Dassault iSight is doing?';
\end{lstlisting}

\item[587:]
\begin{lstlisting}[language=SQL]
SELECT (SELECT AVG(ViewCount)
        FROM posts
        WHERE Tags LIKE '%<humor>%') AS AvgViewCount,
       posts.Title, comments.Text
FROM posts
LEFT JOIN comments ON comments.PostId = posts.Id
WHERE posts.Tags LIKE '%<humor>%';
\end{lstlisting}

\item[598:]
\begin{adjustbox}{max width=\textwidth}
\begin{lstlisting}[language=SQL]
SELECT CASE WHEN c0=0 THEN NULL
            ELSE (c1 - c0) * 100.0 / c0 END AS percentage_difference
FROM (
  SELECT SUM(CASE WHEN Name='student' AND strftime('%Y',Date)='2010' THEN 1 ELSE 0 END) AS c0,
         SUM(CASE WHEN Name='student' AND strftime('%Y',Date)='2011' THEN 1 ELSE 0 END) AS c1
  FROM badges
);
\end{lstlisting}
\end{adjustbox}

\item[639:]
\begin{lstlisting}[language=SQL]
SELECT 100.0 * SUM(CASE WHEN posts.Tags LIKE '%<r>%' THEN 1 ELSE 0 END)
       / COUNT(*) AS percentage
FROM posts
WHERE posts.OwnerDisplayName = 'Community';
\end{lstlisting}

\item[701:]
\begin{lstlisting}[language=SQL]
SELECT 100.0 * SUM(CASE WHEN posts.Score > 50 THEN 1 ELSE 0 END)
       / NULLIF(COUNT(*),0) AS percentage
FROM posts
WHERE posts.OwnerUserId = (
  SELECT users.Id
  FROM users
  ORDER BY users.Reputation DESC
  LIMIT 1
);
\end{lstlisting}

\item[789:]
\begin{lstlisting}[language=SQL]
SELECT AVG(superhero.weight_kg)
FROM superhero
JOIN race ON superhero.race_id = race.id
WHERE race.race = 'Alien';
\end{lstlisting}

\item[793:]
\begin{lstlisting}[language=SQL]
SELECT COUNT(*)
FROM superhero
JOIN race ON superhero.race_id = race.id
JOIN gender ON superhero.gender_id = gender.id
WHERE race.race = 'god/eternal' AND gender.gender = 'male';
\end{lstlisting}

\item[795:]
\begin{lstlisting}[language=SQL]
SELECT COUNT(*)
FROM superhero
JOIN alignment ON superhero.alignment_id = alignment.id
WHERE lower(alignment.alignment) = 'neutral';
\end{lstlisting}

\item[806:]
\begin{lstlisting}[language=SQL]
SELECT colour.colour
FROM superhero
JOIN colour ON superhero.eye_colour_id = colour.id
WHERE superhero.full_name = 'Karen Beecher-Duncan';
\end{lstlisting}

\item[814:]
\begin{lstlisting}[language=SQL]
SELECT DISTINCT colour.colour
FROM superhero
JOIN hero_attribute ON superhero.id = hero_attribute.hero_id
JOIN colour ON superhero.skin_colour_id = colour.id
WHERE hero_attribute.attribute_value = 100;
\end{lstlisting}

\item[817:]
\begin{lstlisting}[language=SQL]
SELECT race.race
FROM superhero
JOIN gender ON superhero.gender_id = gender.id
JOIN colour ON superhero.hair_colour_id = colour.id
JOIN race ON superhero.race_id = race.id
WHERE gender.gender = 'Male' AND colour.colour = 'Blue';
\end{lstlisting}

\item[823:]
\begin{lstlisting}[language=SQL]
SELECT COUNT(*)
FROM superhero
JOIN gender ON superhero.gender_id = gender.id
JOIN publisher ON superhero.publisher_id = publisher.id
WHERE gender.gender = 'Female' AND publisher.publisher_name = 'Marvel Comics';
\end{lstlisting}

\item[829:]
\begin{lstlisting}[language=SQL]
SELECT CASE WHEN dc_count>marvel_count THEN 'DC'
            WHEN dc_count<marvel_count THEN 'Marvel Comics'
            ELSE 'Tie' END AS publisher,
       abs(dc_count-marvel_count) AS difference
FROM (
  SELECT (SELECT COUNT(*)
          FROM superhero s
          JOIN publisher p ON s.publisher_id=p.id
          WHERE p.publisher_name='DC') AS dc_count,
         (SELECT COUNT(*)
          FROM superhero s2
          JOIN publisher p2 ON s2.publisher_id=p2.id
          WHERE p2.publisher_name='Marvel Comics') AS marvel_count
);
\end{lstlisting}

\item[834:]
\begin{lstlisting}[language=SQL]
SELECT 100.0 * SUM(CASE WHEN gender.gender = 'Female' THEN 1 ELSE 0 END)
       / COUNT(*)
FROM superhero
JOIN publisher ON superhero.publisher_id = publisher.id
LEFT JOIN gender ON superhero.gender_id = gender.id
WHERE publisher.publisher_name = 'George Lucas';
\end{lstlisting}

\item[835:]
\begin{adjustbox}{max width=\textwidth}
\begin{lstlisting}[language=SQL]
SELECT AVG(CASE WHEN alignment.alignment='good' THEN 1.0 ELSE 0 END)*100 AS percentage
FROM superhero
JOIN publisher ON superhero.publisher_id = publisher.id
LEFT JOIN alignment ON superhero.alignment_id = alignment.id
WHERE publisher.publisher_name='Marvel Comics';
\end{lstlisting}
\end{adjustbox}

\item[1470:]
\begin{adjustbox}{max width=\textwidth}
\begin{lstlisting}[language=SQL]
SELECT COUNT(DISTINCT gasstations.GasStationID)
FROM gasstations
JOIN transactions_1k ON gasstations.GasStationID = transactions_1k.GasStationID
JOIN products ON transactions_1k.ProductID = products.ProductID
WHERE gasstations.Country = 'CZE' AND products.Description = 'Premium';
\end{lstlisting}
\end{adjustbox}

\item[1476:]
\begin{adjustbox}{max width=\textwidth}
\begin{lstlisting}[language=SQL]
SELECT SUM(CASE WHEN customers.Currency='CZK' THEN yearmonth.Consumption ELSE 0 END)
     - SUM(CASE WHEN customers.Currency='EUR' THEN yearmonth.Consumption ELSE 0 END) AS difference
FROM yearmonth
JOIN customers ON yearmonth.CustomerID=customers.CustomerID
WHERE yearmonth.Date LIKE '2012%';
\end{lstlisting}
\end{adjustbox}

\item[1477:]
\begin{lstlisting}[language=SQL]
SELECT strftime('%Y', transactions_1k.Date) AS Year
FROM transactions_1k
JOIN customers ON transactions_1k.CustomerID = customers.CustomerID
WHERE customers.Currency = 'EUR'
GROUP BY Year
ORDER BY SUM(transactions_1k.Amount) DESC
LIMIT 1;
\end{lstlisting}

\item[1478:]
\begin{lstlisting}[language=SQL]
SELECT customers.Segment
FROM yearmonth
JOIN customers ON yearmonth.CustomerID = customers.CustomerID
GROUP BY customers.Segment
ORDER BY SUM(yearmonth.Consumption) ASC
LIMIT 1;
\end{lstlisting}

\item[1486:]
\begin{lstlisting}[language=SQL]
SELECT CASE WHEN SUM(CASE WHEN Currency='Czech koruna' THEN 1 ELSE 0 END) >
                 SUM(CASE WHEN Currency='euro' THEN 1 ELSE 0 END)
            THEN 'true' ELSE 'false' END AS more_czech,
       SUM(CASE WHEN Currency='Czech koruna' THEN 1 ELSE 0 END) -
       SUM(CASE WHEN Currency='euro' THEN 1 ELSE 0 END) AS difference
FROM customers
WHERE Segment='SME';
\end{lstlisting}

\item[1490:]
\begin{lstlisting}[language=SQL]
SELECT (SELECT COUNT(DISTINCT yearmonth.CustomerID)
        FROM yearmonth
        JOIN customers ON yearmonth.CustomerID=customers.CustomerID
        WHERE customers.Segment='LAM' AND yearmonth.Consumption>46.73)*100.0
        /(SELECT COUNT(DISTINCT CustomerID)
          FROM customers
          WHERE Segment='LAM');
\end{lstlisting}

\item[1501:]
\begin{adjustbox}{max width=\textwidth}
\begin{lstlisting}[language=SQL]
SELECT DISTINCT gasstations.Country
FROM gasstations
JOIN transactions_1k ON gasstations.GasStationID = transactions_1k.GasStationID
WHERE transactions_1k.Date LIKE '2013-06%';
\end{lstlisting}
\end{adjustbox}

\item[1516:]
\begin{lstlisting}[language=SQL]
SELECT COUNT(*)
FROM transactions_1k
JOIN customers ON transactions_1k.CustomerID = customers.CustomerID
WHERE customers.Currency = 'CZK'
  AND transactions_1k.Date = '2012/8/26'
  AND transactions_1k.Time < '12:00:00';
\end{lstlisting}

\item[1526:]
\begin{lstlisting}[language=SQL]
SELECT (t.m2012 - t.m2013) / t.m2012 AS decrease_rate
FROM (
  SELECT MAX(CASE WHEN Date='2012' THEN Consumption END) AS m2012,
         MAX(CASE WHEN Date='2013' THEN Consumption END) AS m2013
  FROM yearmonth
  WHERE CustomerID = (
    SELECT CustomerID
    FROM transactions_1k
    WHERE Date='2012/8/25' AND Amount = 634.8
    LIMIT 1
  )
) AS t;
\end{lstlisting}

\item[1531:]
\begin{adjustbox}{max width=\textwidth}
\begin{lstlisting}[language=SQL]
SELECT customers.CustomerID,
       SUM(transactions_1k.Price) AS total_spent,
       SUM(transactions_1k.Price)/SUM(transactions_1k.Amount) AS avg_price_per_item,
       customers.Currency
FROM transactions_1k
JOIN customers ON transactions_1k.CustomerID=customers.CustomerID
GROUP BY customers.CustomerID, customers.Currency
ORDER BY total_spent DESC
LIMIT 1;
\end{lstlisting}
\end{adjustbox}

\end{enumerate}


\subsection{Methodology}

\subsubsection{Model}
The model used for this assignment was GPT-5-mini. A closed-weight model from OpenAI. I selected this model because of its lightweight, and thus resource efficient, nature, along with the fact that it is among the newest and therefore most minimally tested OpenAI models. All settings on the model were left default, partially because GPT-5-mini does not offer control over settings like temperature or Top-p, and partially to test the out of the box effectiveness of this model. At default settings the model has the \texttt{reasoning\_effort} set to minimal and \texttt{verbosity} set to medium.


\subsubsection{Code}
The SQL predictions werer generated by the \texttt{sql\_generation.mjs} file at the root of my assignment repository. This file does the following:
\begin{enumerate}
\item Parses the subset of queries that I selected from the \texttt{input\_data/queries.json} file. 
\item Loops through each query and for each:
\begin{enumerate}
\item Finds the schema associated with the database that the query is used on.
\item Translates the schema into columns and rows.
\item Queries the model with a list of rules, the formatted schema, and the current query.
\end{enumerate}
\item Prints the model's response to the \texttt{evaluation\_repo/sql\_result/predictions.json} file.
\end{enumerate}

\subsubsection{Rules}
Of the variety of rulesets tested, the following produced the most viable results:
\begin{verbatim}
RULES:
1. Use ONLY the provided table and column names.
2. Use ONLY names that appear in the schema description (no synonyms).
3. Do not invent new tables, columns, or values.
4. Output must be a single line with no explanations or non-SQL text.
5. Prefer the simplest correct form that exactly matches the question intent.
\end{verbatim}


\section{Evaluation}
\subsubsection{Methodology}
The official evaluation scripts in the \texttt{mini\_dev} repository were tailored around use for a specific subset of 500 queries from the dev set. Because of this, I had to generate my own \texttt{mini\_dev\_sqlite.jsonl} and \texttt{mini\_dev\_sqlite\_gold.sql} files using the correct SQL associated with each of the natural language questions I selected. They can be found in \texttt{evaluation\_repo/evaluation/sqlite}. There should also be the complete \texttt{dev\_databases} file there which cannot be committed to github due to it's sheer size. This has been locally added and placed in .gitignore on both of my personal devices used for this assignment. These changes were made along with additional logging used to identify issues during the evaluation step.

\subsubsection{Results}
Three metrics were used to evaluate the LLMs SQL queries: 
\begin{enumerate}
\item Execution Accuracy (EX) - Whether the LLMs query has the same output as the gold query.
\item Reward Value Execution Score (R-VES) - Tests both the correctness and the efficiency relative to the gold query. R-VES was run with 10 iterations per CPU on 16 CPUs simultaneously. Each iteration had a time-out of 2 seconds.
\item Soft-F1 (EX) - Measures the similarity between the LLMs queries and the gold queries regardless of correctness.
\end{enumerate}
Errors in SQL queries and failures to execute are simply treated as inaccuracies on the part of the LLM, and detract from both the EX and R-VES without halting the running of tests.

\begin{table}[h!]
\centering
\caption{Evaluation results across difficulty levels.}
\begin{tabular}{lcccc}
\hline
\textbf{Metric} & \textbf{Simple (\%)} & \textbf{Moderate (\%)} & \textbf{Challenging (\%)} & \textbf{Total (\%)} \\
\hline
EX      & 35.00 & 37.50 & 28.57 & 34.00 \\
R-VES   & 17.42 & 17.08 & 21.43 & 18.43 \\
Soft-F1 & 35.03 & 44.05 & 25.00 & 35.11 \\
\hline
\end{tabular}
\end{table}


\section{Analysis}
\subsection{Visual Trends}
From simply inspecting the gold SQL file beside the LLM's predictions, it is apparent that the predicted queries are generally longer and more complex than the gold queries, despite the instructions to the model being very clear about keeping answers as simple as possible. This could contribute to the R-VES being significantly lower than the EX and Soft-F1, as many of the queries are technically correct, but run slowly. This is likely caused by the default model verbosity being set to medium. 

\subsection{Difficulty Comparison}
Strangely, difficulty had a more minimal impact on accuracy than expected. One would theorize that simpler questions would be easier for the model to solve, but the EX and Soft-F1 values are highest for moderate difficulties. The R-VES is even stranger, being highest for the challenging queries. This could imply that while the model's solutions vary heavily from the gold SQL,  the optimal queries for the challenging questions are slower, and thus the unique solutions concocted by the model have a reduced difference in speed compared to the simple queries where the gold is much quicker. 

\section{Comparison}

\begin{table}[h]
\centering
\begin{tabular}{lcccc}
\hline
\textbf{Model / Method} & \textbf{Simple} & \textbf{Moderate} & \textbf{Challenging} & \textbf{Total} \\
\hline
\multicolumn{5}{l}{\textit{Execution Accuracy (EX, \%)}} \\
\hline
GPT-5-mini (Selected Subset)         & 35.00 & 37.50 & 28.57 & 34.00 \\
SuperSQL (BIRD dev)           & 66.90    & 46.50    & 43.80   & 58.50 \\
RSL-SQL + DeepSeek (BIRD dev)         & 69.73    & 54.09    & 54.48    & 63.56 \\
\hline
\multicolumn{5}{l}{\textit{Valid Efficiency Score / R-VES (\%)}} \\
\hline
GPT-5-mini (Selected Subset)         & 17.42 & 17.08 & 21.43 & 18.43 \\
SuperSQL (BIRD dev)           & 69.75    & 50.55    & 49.08    & 61.99 \\
RSL-SQL + DeepSeek (BIRD dev)         & --    & --    & --    & 67.68 \\
\hline
\multicolumn{5}{l}{\textit{Soft-F1 (\%)}} \\
\hline
GPT-5-mini (Selected Subset)          & 35.03 & 44.05 & 25.00 & 35.11 \\
\hline
\end{tabular}
\caption{Comparison of evaluation results with GPT-5-mini in this assignment on a 50 query subset with RSL-SQL + Deepseek [1] and SuperSQL [2] on the BIRD development set.}
\end{table}

The SuperSQL model and combination of the RSL-SQL technique with the Deepseek both acheived scores for both accuracy and efficiency significantly above what GPT-5-mini accomplished using the techniques presented in this paper. Both papers for these approaches involved signficant training, while this attempt only involved prompting and the full schema being included with each question. Additionally, RSL-SQL involved schema pruning, SuperSQL utilized post-processing, and both pepers used schema linking to further optimize the model's performance. All of these techniques are very effective ways of improving the model's understanding of the database associated with a given query.

Both of these papers were assessed against the entire dev dataset which includes over 1500 queries, whereas this report covered only 50. This smaller sample may have affected the average difficulty of queries either positively or negatively for this assignment. 

This report involves the analysis of the responses' Soft-F1 scores, a unique metric not employed by this previous research. This provides valuable insight outside of the scope of what is covered in these papers. 

\section{Resources}

\begin{itemize}
\item OpenAI documentation was used to aid in the implementation of the GPT-5-mini API into the code [3].
\item All evaluation code was either taken directly or modified slightly after being taken directly from the \texttt{bird\_bench/mini\_dev} repository on GitHub
\item ChatGPT was used to assist in error identification while programming, altering evaluation code, and for translating JSON files and raw data into lists and tables for this report.
\end{itemize}

\begin{thebibliography}{9}

\bibitem{cao2024rslsql}
Z. Cao, Y. Zheng, Z. Fan, X. Zhang, W. Chen, and X. Bai, 
``RSL-SQL: Robust Schema Linking in Text-to-SQL Generation,'' 
\textit{arXiv preprint arXiv:2411.00073}, 2024.

\bibitem{li2024bird}
Boyan Li, Yuyu Luo, Chengliang Chai, Guoliang Li, Nan Tang. The Dawn
of Natural Language to SQL: Are We Fully Ready?. PVLDB, 17(11): 3318 -
3331, 2024.
doi:10.14778/3681954.3682003

\bibitem{openai}
OpenAI API Documentation. Retrieved Sept. 2025 from https://platform.openai.com/docs/quickstart 

\bibitem{bird_dev}
Bird-Bench. Bird-bench/mini\_dev. Retrieved Sept. 2025 from https://github.com/bird-bench/mini\_dev 

\bibitem{chatgpt}
OpenAI, ChatGPT 5 \textit{ https://chat.openai.com/chat, 2025}

\end{thebibliography}


\end{document}
